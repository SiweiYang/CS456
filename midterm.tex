% Exam Template for UMTYMP and Math Department courses
%
% Using Philip Hirschhorn's exam.cls: http://www-math.mit.edu/~psh/#ExamCls
%
% run pdflatex on a finished exam at least three times to do the grading table on front page.
%
%%%%%%%%%%%%%%%%%%%%%%%%%%%%%%%%%%%%%%%%%%%%%%%%%%%%%%%%%%%%%%%%%%%%%%%%%%%%%%%%%%%%%%%%%%%%%%

% These lines can probably stay unchanged, although you can remove the last
% two packages if you're not making pictures with tikz.
\documentclass[11pt]{exam}
\RequirePackage{amssymb, amsfonts, amsmath, latexsym, verbatim, xspace, setspace}
\RequirePackage{tikz, pgflibraryplotmarks}

% By default LaTeX uses large margins.  This doesn't work well on exams; problems
% end up in the "middle" of the page, reducing the amount of space for students
% to work on them.
\usepackage[margin=1in]{geometry}


% Here's where you edit the Class, Exam, Date, etc.
\newcommand{\class}{CS 456}
\newcommand{\term}{Spring 2013}
\newcommand{\examnum}{Exam 001}
\newcommand{\examdate}{today}
\newcommand{\timelimit}{90 Minutes}

% For an exam, single spacing is most appropriate
\singlespacing
% \onehalfspacing
% \doublespacing

% For an exam, we generally want to turn off paragraph indentation
\parindent 0ex

\begin{document} 

% These commands set up the running header on the top of the exam pages
\pagestyle{head}
\firstpageheader{}{}{}
\runningheader{\class}{\examnum\ - Page \thepage\ of \numpages}{\examdate}
\runningheadrule

\begin{flushright}
\begin{tabular}{p{2.8in} r l}
\textbf{\class} & \textbf{Name (Print):} & \makebox[2in]{\hrulefill}\\
\textbf{\term} &&\\
\textbf{\examnum} &&\\
\textbf{\examdate} &&\\
\textbf{Time Limit: \timelimit} & Teaching Assistant & \makebox[2in]{\hrulefill}
\end{tabular}\\
\end{flushright}
\rule[1ex]{\textwidth}{.1pt}


This exam contains \numpages\ pages (including this cover page) and
\numquestions\ problems.  Check to see if any pages are missing.  Enter
all requested information on the top of this page, and put your initials
on the top of every page, in case the pages become separated.\\

You may \textit{not} use your books, notes, or any calculator on this exam.\\

You are required to show your work on each problem on this exam.  The following rules apply:\\

\begin{minipage}[t]{3.7in}
\vspace{0pt}
\begin{itemize}

\item \textbf{If you use a ``fundamental theorem'' you must indicate this} and explain
why the theorem may be applied.

\item \textbf{Organize your work}, in a reasonably neat and coherent way, in
the space provided. Work scattered all over the page without a clear ordering will 
receive very little credit.  

\item \textbf{Mysterious or unsupported answers will not receive full
credit}.  A correct answer, unsupported by calculations, explanation,
or algebraic work will receive no credit; an incorrect answer supported
by substantially correct calculations and explanations might still receive
partial credit.


\item If you need more space, use the back of the pages; clearly indicate when you have done this.
\end{itemize}

Do not write in the table to the right.
\end{minipage}
\hfill
\begin{minipage}[t]{2.3in}
\vspace{0pt}
%\cellwidth{3em}
\gradetablestretch{2}
\vqword{Problem}
\addpoints % required here by exam.cls, even though questions haven't started yet.	
\gradetable[v]%[pages]  % Use [pages] to have grading table by page instead of question

\end{minipage}
\newpage % End of cover page

%%%%%%%%%%%%%%%%%%%%%%%%%%%%%%%%%%%%%%%%%%%%%%%%%%%%%%%%%%%%%%%%%%%%%%%%%%%%%%%%%%%%%
%
% See http://www-math.mit.edu/~psh/#ExamCls for full documentation, but the questions
% below give an idea of how to write questions [with parts] and have the points
% tracked automatically on the cover page.
%
%
%%%%%%%%%%%%%%%%%%%%%%%%%%%%%%%%%%%%%%%%%%%%%%%%%%%%%%%%%%%%%%%%%%%%%%%%%%%%%%%%%%%%%

\begin{questions}
%%%%%%%%%%%%%%%%%%%%%%%%%%%%%%%%%%%%%%%%%%%%%%%%%%%%%%%%%%%%%%%%%%%%%%%%%%%%%%%%%%%%%%%%%%%%%%%%%%%%%%%%%%%%
\newpage
\addpoints
\question General
\begin{parts}
\part[4] Give 2 reasons for using a layered protocol architecture.
\vfill
\part[8] List and breifly explain the four delay components involved in transmitting a message over a packet-switched network? Make sure to explain where the respective delay can occur.
\vspace{3.5in}
\part[6] Pick two delay components and construct two application examples to illustrate the significantly different impact of each component.
\vspace{3.5in}
\part[4] Telephony requires a strict maximum end-to-end delay, assume 150 milliseconds. Also assume a transmission rate of 64 kbits/sec. In the view of requirements, should an internet telephony application use a packet size of 100 or 1000 bytes? Why?
\vfill
\part[2] What is meant when the control signalling if FTP is characterized as "out of band"?
\vfill
\end{parts}
%%%%%%%%%%%%%%%%%%%%%%%%%%%%%%%%%%%%%%%%%%%%%%%%%%%%%%%%%%%%%%%%%%%%%%%%%%%%%%%%%%%%%%%%%%%%%%%%%%%%%%%%%%%%
\newpage
\addpoints
\question Application Layer
\begin{parts}
\part[4] The content length field in HTTP is optional in non-persistent connections and mandatory in persistent connections. Why?
\vspace{2.5in}
\part[4] Describe how cookies can be used by a web server to maintain a use profile.
\vspace{2.5in}
\part[4] Explain how a web server can be regarded as stateless or as stateful server. Give at least one example each.
\vspace{1.5in}
\part[2] What is the primary machenism thay DNS uses to reduce the number of messages being exchanges between DNS servers?
\vfill
\part[4] Explain the difference between a local name server and an authoritative name server?
\vspace{2.5in}
\part[4] Explain the role of CNAME entries in DNS. There is similar functionality in file system. What is it?
\vspace{2.5in}
\end{parts}
%%%%%%%%%%%%%%%%%%%%%%%%%%%%%%%%%%%%%%%%%%%%%%%%%%%%%%%%%%%%%%%%%%%%%%%%%%%%%%%%%%%%%%%%%%%%%%%%%%%%%%%%%%%%
\newpage
\addpoints
\question Transport Layer
\begin{parts}
\part[2] What is the purpose of port number in the transport layer?
\vfill
\part[2] What is the purpose of the checksum in the transport layer?
\vfill
\part[2] Assume a network service guarantees that packets are never reordered. Ignoring any performance implications, does a reliable tranport layer still have to employ sequence numbers?
\vfill
\part[2] What is the motivation for a reliable data transfer protocol to support pipelining?
\vfill
\part[2] A transport protocol uses 4 bits for sequence numbers. What is the maximum window size, if the transport protocol uses Go-Back-N. What is it uses Selective Repeat?
\vfill
\end{parts}
%%%%%%%%%%%%%%%%%%%%%%%%%%%%%%%%%%%%%%%%%%%%%%%%%%%%%%%%%%%%%%%%%%%%%%%%%%%%%%%%%%%%%%%%%%%%%%%%%%%%%%%%%%%%
\addpoints
\question Transport Protocol
\begin{parts}
\part[2] Why does TCP transmit the source port number in the header of data segments?
\vspace{4.5in}
\part[2] Why does UDP transmit the source port number in the header of data segments?
\vspace{4.5in}
\part[2] TCP has length of header field while UDP does not. Why?
\vspace{4.5in}
\part[9] Describe the functionality and usage of socket interface at a server using connection-oriented and reliable(aka TCP) communication with multiple consecutive clients. Use psedo-code or just name and explain the different steps that server carries out.
\vspace{4.5in}
\end{parts}
%%%%%%%%%%%%%%%%%%%%%%%%%%%%%%%%%%%%%%%%%%%%%%%%%%%%%%%%%%%%%%%%%%%%%%%%%%%%%%%%%%%%%%%%%%%%%%%%%%%%%%%%%%%%
\addpoints
\question TCP and Congestion Control
\begin{parts}
\part[8] Consider a use who accesses the internet using a connection with link speed of 1Mbps. Consider only the latency of TCP connection to download a web page. Assume the round trip time between the host and the web server is 12ms; the size of web page is 13Kbytes; MSS of TCP is 1Kbytes; and no packets are lost. For simplicity, you can ignore the transmission time of all short packets(e.g. ACK, HTTP request), packet headers, and overhead of lower layers. For ease of calculation, you can assume 1K = 1000, and 1M = 1000,000. How long does it take until all web page data is transmitted to client? Present details of your calculation to earn partial marks, if only some of your intermediate results are correct.
\vspace{4.5in}
\end{parts}
%%%%%%%%%%%%%%%%%%%%%%%%%%%%%%%%%%%%%%%%%%%%%%%%%%%%%%%%%%%%%%%%%%%%%%%%%%%%%%%%%%%%%%%%%%%%%%%%%%%%%%%%%%%%
\addpoints
\question Network Layer
\begin{parts}
\part[6] Describe the forwarding table of a router in virtual circuit network. What fields are stored in a row? Which fields form the key? Are any fields optional
\vspace{4.5in}
\part[3] Name two conditions that need to be satisfied, so that a router in a virtual circuit network can use a compact array of K entries as forwarding table, independent of other router's algorithm.
\vfill
\part[2] Does it make sense for a high-speed virtual circuit router to use a central forwarding table for all interfaces? Briefly explain your answer.
\vfill
\part[2] Which nodes perform IP reassembly
\vfill
\end{parts}


\end{questions}
\end{document}
